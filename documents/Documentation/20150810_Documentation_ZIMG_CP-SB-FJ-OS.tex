% % % % % % % % % % % % % % % % % % % % % %
%
%	PREAMBLE
%
% % % % % % % % % % % % % % % % % % % % % %

% PAGE STYLE SETTINGS
	%\documentclass[12pt, oneside, a4paper]{art}		% PDF report style
	\documentclass[12pt, oneside, a4paper]{article}		% PDF article style
	%\documentclass[12pt, twoside, a4paper]{report}		% book style

% PACKAGES SETTINGS
	\usepackage[T1]{fontenc} 							% encoding input T1 codec
	\usepackage[utf8]{inputenc}							% encoding output
	\usepackage[ngerman]{babel}							% language settings German
	%\usepackage[english]{babel}						% language settings English
	\usepackage{fancyhdr}								% custom header and footer
	\usepackage{lastpage}								% label (variable) {LastPage}
	\usepackage[square,sort,comma,numbers]{natbib}		% reference style
	\usepackage{cite}									% citations
	\usepackage{listings}								% integration of listings
	\usepackage{graphicx}								% integration of graphics
	\usepackage{blindtext}								% generate bt with \blindtext or \blindtext[AMOUNT]
	\usepackage{microtype}								% justification
	\usepackage{lettrine}								% initial letter big
	\usepackage{multicol}								% multi column
	\usepackage{float}									% for graphics with caption
	\usepackage{seqsplit}								% for breaks in long titles
	\usepackage[a4paper, bindingoffset=.508cm, left=2.54cm, right=2.54cm, top=2.54cm, bottom=2.54cm, footskip=.63cm]{geometry}												% for customized margins

% FONTS SETTINGS
	%\usepackage[sfdefault]{cabin}						% default font
	%\usepackage[default]{droidserif}					% default font
	\usepackage{courier}								% font for listings


% HEADER AND FOOTER SETTINGS
	\setlength{\headheight}{0.5291cm}
	\renewcommand{\headrulewidth}{0.01cm}
	\pagestyle{fancy}
	\lhead{\insertauthor}								% top left
	\chead{}											% top center
	\rhead{\inserttitle}								% top right
	\lfoot{August 2015}									% bottom left
	\cfoot{}											% bottom center
	\rfoot{\thepage}									% bottom right
	%\rfoot{Page \thepage \ of \pageref{LastPage}}		% bottom right

% LISTINGS SETTINGS
	\lstset{
		basicstyle=\footnotesize\ttfamily,  			% set font Courier
		captionpos=b, 									% description under listings
		xleftmargin=0.5cm								% indent listing block
	}

% REFERENCES SETTINGS
	\addto\captionsenglish{\renewcommand{\refname}{References}}
	\bibliographystyle{plain}							% bib style

% TITLE SETTINGS										
	\title{ZIMG}
	\author{Christian Pleines, Stefan Bieliauskas, Fabian Junge, Oke Schwien}
	\date{2015 08 10}

\makeatletter
\renewcommand{\maketitle}{\bgroup\setlength{\parindent}{0pt}
	\begin{center}
		\vspace{4cm}
  		\seqsplit{\Huge{\textbf{\inserttitle}}}

  		\vspace{4cm}
  		\large{\insertauthor}

  		\vspace{2cm}
  		\small{Hochschule Bremen -- City University of Applied Sciences}

  		\small{Datenbankbasierte Web-Anwendungen, \insertdate}
	\end{center}\egroup
}

\let\inserttitle\@title

\let\insertauthor\@author

\let\insertdate\@date

\makeatother

% TEXT SETTINGS
	\setlength{\columnsep}{0.8818cm}					% column seperation
	\setlength{\parindent}{0cm}							% indentation off -> 0cm

% % % % % % % % % % % % % % % % % % % % % %
%
%	DOCUMENT
%
% % % % % % % % % % % % % % % % % % % % % %

\begin{document}

% TITLE
\begin{figure}[H]
	\centering
 	\includegraphics[width=6cm]{footage/Hochschule_Bremen_Logo_RGB} 
	\label{logoofhochschulebremengermany}
\end{figure}

\maketitle												% print default title
\thispagestyle{empty}									% disable page numbers

\newpage

% INHALTSVERZEICHNIS
\tableofcontents
%\thispagestyle{empty}									% disable page numbers

\newpage

% TEXT
\section{Spezifikation}
% 1 Spezifikation
% In diesem Kapitel sollen die folgenden Punkte erläutert werden:
% 	- Was leistet die Webanwendung?
%	- Welchen Mehrwert bietet die Webanwendung für dessen Nutzer?
Im Rahmen des Moduls Datenbankbasierte Web-Anwedungen an der Hochschule Bremen im Sommersemester 2015 bei Steve Liedtke entstand das Web-Imageboard ZIMG. ZIMG ist eine Plattform zum Austausch von Bildern und Gifs über das Internet. Zur Zielgruppe für diese Plattform gehören diejenigen Studierenden, die der Fakultät 4 angehören und den Standort ZIMT (Zentrum für Informatik und Medientechnologien) besuchen. \\
Die Studierenden können sich einloggen (WENN EINGETRAGEN) und Bilder auf ZIMG hochladen und diese so mit der Community rund um ZIMG teilen. Zusätzlich können hochgeladene Bilder von den Benutzern favorisiert, durch Tags kategorisiert und kommentiert werden. \\
ZIMG bietet außerdem die Möglichkeit die Top Ten Tags, Top Ten Bilder und Top Five User zu begutachten. \\
Diese Funktionen machen ZIMG zu einem Ort des Austausches zwischen Studierenden, die beispielsweise ihre Erlebnisse im Semester oder Auslandssemester mit ihren Kommilitonen teilen wollen.

Einloggen (nicht registrieren) > uploaden > favorisieren > taggen > kommentieren > Top Ten bilder > top ten tags > top 5 uploader

\subsection{Anforderungen}
% 1.1 Anforderungen
% In diesem Unterkapitel sollen die vorgesehenen und die tatsächlich umgesetzten Anforderungen beschrieben und mit Hilfe von Use-Case- und Aktivitätsdiagrammen konkretisiert werden.
\blindtext

\section{Architektur}
% 2 Architektur
% In diesem Kapitel soll beschrieben werden, wie das Projekt und die Datenbank strukturiert ist und welche Technologien und Frameworks verwendet werden.
\blindtext

\subsection{Gesamtarchitektur}
% 2.1 Gesamtarchitektur
% In diesem Unterkapitel soll beschrieben werden, wie die Gesamtarchitektur der entwickelten Webanwendung aussieht. Dies kann mit UML Paket- oder Klassendiagrammen unterstützt werden.
\blindtext

\subsection{Programmabläufe}
% 2.2 Programmabläufe
% In diesem Unterkapitel sollte exemplarisch gezeigt werden, wie die typischen Abläufe in der Webanwendung aussehen. Hierfür sollten Aktivitäts-/Sequenzdiagramme genutzt werden.
\blindtext

\subsection{Persistenz}
% 2.3 Persistenz
% In diesem Unterkapitel soll die genutzte Datenbank und der Zugriff auf ihre Daten erläutert werden.
\blindtext

\subsubsection{Konzeption}
% 2.3.1 Konzeption
% In diesem Unterkapitel soll die konzepierte Datenbank nach ER-Diagramm beschrieben werden
\blindtext

\subsubsection{Umsetzung}
% 2.3.2 Umsetzung
% In diesem Unterkapitel soll die Datenbank nach relationalem Modell beschrieben werden. Außerdem sollten anwendungsspezifische Constraints aufgelistet werden, die bei der Implementierung bedacht werden müssen.
\blindtext

\subsubsection{Zugriff}
% 2.3.3 Zugriff
% In diesem Unterkapitel soll beschrieben werden mit welchen Technologien und Frameworks von der Webanwendung auf die Datenbank zugegriffen wird.
\blindtext

\subsection{Präsentation}
% 2.4 Präsentation
% In diesem Unterkapitel soll die Präsentationsschicht der Webanwendung beschrieben werden. Hierzu u.a. die genutzte Technologie/Framework. 
\blindtext

\section{Installation}
% 3 Installation
% In diesem Kapitel soll beschrieben werden, welche Programmierwerkzeuge verwendet wurden und wie die Webanwendung lokal aufgesetzt werden kann
\blindtext

\subsection{Programmierwerkzeuge}
% 3.1 Programmierwerkzeuge
% Auflistung der verwendeten Programmierwerkzeugen mit Vor- und Nachteilen
\blindtext

\subsection{Installation der Entwicklungsumgebung}
% 3.2 Installation der Entwicklungsumgebung
% Detaillierte Erklärung welche Maßnahmen notwendig sind, um die Entwicklungsumgebung aufzusetzen, sodass an dem Projekt mitentwickelt werden kann 
\blindtext

\subsection{Konfiguration der Datenbank}
% 3.4 Konfiguration der Datenbank
% Beschreibung welche Schritte durchgeführt werden müssen, um die Datenbank mitsamt ihrer Struktur und Beispieldaten auf einem lokalen System aufzusetzen
\blindtext

\subsection{Installation der Anwendung}
% 3.3 Installation der Anwendung
% Beschreibung wie die Anwendung auf dem lokalen Rechner gestartet werden kann 
\blindtext

\end{document}

% % % % % % % % % % % % % % % % % % % % % %
%	EXAMPLES
% % % % % % % % % % % % % % % % % % % % % %

% initial letter big, footnotes, dots
	% \lettrine[lines=2]{H}{}ello\footnote{Hello is a word and it is a nice word; in other languages hello means Hallo, Bonjour, Hoi, \ldots} you, here I am the god and you are the ant.

% different lists
	% \begin{itemize}
	% 	\item one item
	% 	\item another item
	% 	\begin{itemize}
	% 		\item one subitem
	% 		\item another subitem
	% 	\end{itemize}
	% \end{itemize}

	% \begin{description}
	%   \item[girls] Sabrina, Monica, Suzanne
	%   \item[boys] Peter, Jack, Will
	%   \item[animals] lion, elephant, monkey
	% \end{description}

	% \begin{itemize}
	% 	\item one item
	% 	\item another item
	% 	\begin{enumerate}
	% 		\item first subitem
	% 		\item second subitem
	% 	\end{enumerate}
	% \end{itemize}

% include dividers
	% \begin{flushleft}
	% 	\rule{\linewidth}{0.01cm}
	% \end{flushleft}

% include graphics with caption and label
	% \begin{figure}[H]
	% 	\centering
	% 	\includegraphics[width=\linewidth]{footage/Hochschule_Bremen_Logo} 
	% 	\caption{Logo of Hochschule Bremen, Germany}
	%	\label{logoofhochschulebremengermany}
	% \end{figure}
  % how to mention this graphic in text
	% texttextext (Fig. \ref{logoofhochschulebremengermany}) texttexttext.

% quotation -- leading quote
	% \begin{quote}
	% \centering Our goal is to design everything so it's beautifully simple. \\
	% -- Larry Page, Google's July 2013 Q2 earnings call
	% \end{quote}

% tabular
%	\begin{table}[h!]
%	\centering
%	\scalebox{1.0}{
%	\begin{tabular}{| p{4cm} | p{10.3cm} | l}
%	\hline                       
%	\textbf{Taste} 	& \textbf{Befehl}		\\
%	\hline
%	A				& Bewegt den SuperJumperMan nach links	\\
%	\hline
%	D				& Bewegt den SuperJumperMan nach rechts	\\
%	\hline
%	Leertaste		& Lässt den SuperJumperMan springen \\
%	\hline
%	Pfeiltaste Oben & Zoomt die Kamera herein \\
%	\hline
%	Pfeiltaste Unten & Zoomt die Kamera heraus \\
%	\hline
%	Pfeiltaste Links & Dreht die Kamera nach links \\
%	\hline
%	Pfeiltaste Rechts & Dreht die Kamera nach rechts \\
%	\hline
%	\end{tabular}
%	}
%	\caption{Steuerung von SuperJumperManSupreme}
%	\label{Steuerung}
%	\end{table}